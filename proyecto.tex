\documentclass{article}
\usepackage[spanish]{babel}
\usepackage[utf8]{inputenc}

\begin{document}

\title{Estudio de la curva de rotación y la masa no bariónica de la galaxia UGC11916}

\author{Iñigo Pérez Mérové-Pierre}

\date{16 de octubre de 2018}

\maketitle

\begin{abstract}
En este trabajo vamos a realizar un estudio de la galaxia con anillo polar UGC1196 a partir de los datos obtenidos de observaciones realizadas con la técnica de espectroscopia de campo integral. Las galaxias con anillo polar, debido a su peculiar morfología y a su origen y evolución, son objetos que nos pueden proporcionar gran información sobre las interacciones galácticas y sobre la naturaleza y distribución de la materia oscura en las galaxias. En primer lugar, realizaremos un análisis morfológico en el continuo de la galaxia, analizando la intensidad del espectro en su zona central. Mediante el estudio del gas ionizado (HII) a través de la línea de $H\alpha$, analizando sus mapas de velocidad, la curva de rotación y comparando con las velocidades y curvas de rotación estelares, podemos detectar la presencia de sistemas cinemáticos diferenciados que nos permitan inferir la presencia de anillos polares en estas galaxias. Estudiaremos la relación entre la masa luminosa ($M_{L}$), esto es, la masa total de la galaxia en función de su luminosidad, y la masa dinámica ($M_{D}$), la masa estimada de la galaxia a partir de su velocidad de rotación, para el gas ionizado como de las estrellas. Una relación $M_{D} > M_{L}$ nos indica la presencia de una masa no bariónica que está manteniendo la galaxia en equilibrio gravitatorio. Finalmente estudiaremos la población estelar de cada galaxia, analizando los mapas de edad estelar, con los que podemos diferenciar las zonas de la galaxia con presencia de estrellas jóvenes con mayor metalicidad (población I) y las zonas donde se encuentra la población más vieja, de baja metalicidad y con edades que alcanzan los 10 Ga (población II). 
\end{abstract}

\end{document}